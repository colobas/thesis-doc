\documentclass{article}

\usepackage[final,nonatbib]{neurips_2018}
\usepackage[utf8]{inputenc} % allow utf-8 input
\usepackage[T1]{fontenc}    % use 8-bit T1 fonts
\usepackage{hyperref}       % hyperlinks
\usepackage{url}            % simple URL typesetting
\usepackage{booktabs}       % professional-quality tables
\usepackage{amsfonts}       % blackboard math symbols
\usepackage{nicefrac}       % compact symbols for 1/2, etc.
\usepackage{microtype}      % microtypography
\usepackage{csquotes}
\renewcommand{\mkbegdispquote}[2]{\itshape}

\title{(Deep) Generative Models for Multi-variate Time-series: 
An industrial application}

\author{%
  Guilherme Pires \\
  Instituto Superior Técnico \\
  Jungle \\
  \texttt{mail@gpir.es} \\
}

\begin{document}

\maketitle

\begin{abstract}
  The abstract paragraph should be indented \nicefrac{1}{2}~inch (3~picas) on
  both the left- and right-hand margins. Use 10~point type, with a vertical
  spacing (leading) of 11~points.  The word \textbf{Abstract} must be centered,
  bold, and in point size 12. Two line spaces precede the abstract. The abstract
  must be limited to one paragraph.
\end{abstract}

\section{Prelude}
\begin{displayquote}
Of all obstacles to a thoroughly penetrating account of existence, none looms
up more dismayingly than "time" (...). Explain time? Not without explaining 
existence. Explain existence? Not without explaining time.
\end{displayquote}

\section{Introduction}

\subsection{Dynamical Systems}
The study of dynamical systems is ubiquitous to a large number of areas of
interest: biological systems, the weather, planetary systems, industrial
systems, social systems, etc. In the era of data, these systems are more
sensorized than ever, creating the opportunity to explore said data and
discover patterns and insights, which can change the way we interact with
these systems, and even enable us to control them.

There are \{multiple,many,some,two\} different types of insights one can try to
extract from data about dynamical systems:
\begin{itemize}
    \item \textbf{Predictions} about the most likely future states of the system,
        given its history.
    \item \textbf{Explanations} about the system's behaviour.
\end{itemize}

Both find their application in different scenarios.

\subsection{Interpretability/Explainability and the Manifold Hypothesis}
Some systems - like industrial systems - are characterized by such a large
number of variables, that even if the system's behaviour were simple, a human
wouldn't be able to identify nor leverage that fact. At the same time, the
type of strategies usually put in place to deal with this type of data are
either heavily-based on domain knowledge (which isn't always available), or
so complex themselves that they become uninterpretable



\bibliographystyle{plain}
\bibliography{../Thesis_bib_DB}

\end{document}
