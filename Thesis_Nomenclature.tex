% The definitions can be placed anywhere in the document body
% and their order is sorted by <symbol> automatically when
% calling makeindex in the makefile
%
% The \glossary command has the following syntax:
%
% \glossary{entry}
%
% The \nomenclature command has the following syntax:
%
% \nomenclature[<prefix>]{<symbol>}{<description>}
%
% where <prefix> is used for fine tuning the sort order,
% <symbol> is the symbol to be described, and <description> is
% the actual description.

% ----------------------------------------------------------------------
% Roman symbols [r]
\nomenclature[ru]{$\bf u$}{Velocity vector.}
\nomenclature[ru]{$u,v,w$}{Velocity Cartesian components.}
\nomenclature[rp]{$p$}{Pressure.}
\nomenclature[rC]{$C_D$}{Coefficient of drag.}
\nomenclature[rC]{$C_L$}{Coefficient of lift.}
\nomenclature[rC]{$C_M$}{Coefficient of moment.}

% ----------------------------------------------------------------------
% Greek symbols [g]
\nomenclature[g]{$\rho$}{Density.}
\nomenclature[g]{$\alpha$}{Angle of attack.}
\nomenclature[g]{$\beta$}{Angle of side-slip.}
\nomenclature[g]{$\mu$}{Molecular viscosity coefficient.}
\nomenclature[g]{$\kappa$}{Thermal conductivity coefficient.}

% ----------------------------------------------------------------------
% Subscripts [s]
\nomenclature[s]{$x,y,z$}{Cartesian components.}
\nomenclature[s]{$i,j,k$}{Computational indexes.}
\nomenclature[s]{$\infty$}{Free-stream condition.}
\nomenclature[s]{ref}{Reference condition.}
\nomenclature[s]{$n$}{Normal component.}

% ----------------------------------------------------------------------
% Supercripts [t]
\nomenclature[t]{T}{Transpose.}
\nomenclature[t]{*}{Adjoint.}

