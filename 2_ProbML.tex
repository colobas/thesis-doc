\chapter{Probabilistic Modelling}
\label{chapter:probmodel}

\section{Introduction}
\label{section:probmodelintro}
Probabilistic modelling can be described as the task of finding the probability
distribution that best describes a given dataset $\mathcal{D}$, i.e. a
distribution $p(\mathcal{D})$ such that $\mathcal{D} \sim p(\mathcal{D})$
is as plausible as possible. Not only that, we normally want the distribution
to be representative of the real process that generated $\mathcal{D}$. That is
to say we want $p(\mathcal{D})$ to generalize to data we haven't yet observed.

There are effectively infinite possible distributions that could have
generated the data we observe, but won't generalize to unobserved data.
Commonly, the first way in which we restrict the set of possible distributions is
to assume that the distribution of interest has a parametric form, and hence is
defined by a set of parameters $\theta$ \footnote{For the type of models and problems
dealt with in this work, I will assume $\theta$ is finite, but it's worth noting
that there are models which have $\theta$ \emph{grow} with the dataset size.
These are called non-parametric models. They come with their own advantages and
disadvantages, which are out of the scope of this work.} .

Intuitively, the size of $\theta$ is deeply connected with the expressiveness
of the model. In practice, this translates to the observation that if we make
the model expressive enough, it can fit the observed data arbitrarily well. Naïvely,
this would be a desirable characteristic. However, it normally comes at the cost
of sacrificing generalization capability (TODO: Explain why). One way to counter
this is to give preference to more parsimonious models - models that use less
parameters and are less complex. There are strategies to make this mathematically
objective and prevent the modeller from both having too complex and too simple
models. Some of those methods are he Bayesian Information Criterion, the Akaike
Information Criterion and the Minimum Description Length.

\section{Structure and Latent Variables}
\label{section:probmodellatvar}
In some cases, one might want to leverage some available domain knowledge. This
often translates into assuming that there is some latent structure in the data.
This structure is encoded into latent variables and their influence over the
observable variables.

In this scenario, we become interested in the distribution given by
$p(x, z, \theta_x, \theta_z)$, where $z$ is the latent variable, $\theta_x$ is
the parameter vector for the distribution over $\mathcal{X}$, and $\theta_z$ is
the parameter vector for the distribution over $\mathcal{Z}$.

For structure and latent variables to be useful we normally make the additional
assumption that we have the ability of factoring that distribution in ways that
make it tractable. One common factorization is:
\begin{align}
    p(x, z, \theta) = p(x| z, \theta_x) p(z | \theta_z) p(\theta_x) p(\theta_z)
\end{align},

\section{Approximate Inference}
\label{section:probmodelinf}
For simplicity, let us consider $\theta$ as part of the latent variables $z$.
This means that the model is simply written as the joint distribution: $p(x, z)$.
\emph{Inference}\footnote{If we hadn't collapsed $\theta$ into $z$ and were instead handling
separately, we would call \textbf{inference} to the task of finding $z$ and
\textbf{learning} to the task of finding $\theta$} is the task of finding the 
most probable $z$ after having observed $x$ . Specifically, the goal is
to find the posterior distribution of $z$, given $x$, i.e.: $p(z|x)$.

Recall Bayes' Law:

\begin{align}
    p(z|x) &= \frac{p(x|z)p(z)}{p(x)} \\
           &= \frac{p(x|z)p(z)}{\int p(x|z')p(z') dz'}
\end{align}

For the vast majory of cases, the integral on the denominator will be
intractable. To overcome this difficulty we normal resort to two families
of methods: Monte-Carlo methods, and Variational methods.

\subsection{Monte-Carlo Methods}
\label{subsection:mcmc}

Monte-Carlo methods work by using sampling techniques to approximate the
intractable integral. The most powerful subclass of these methods is called
Markov-chain Monte-Carlo. Its approach consists of devising a scheme that
allows for sampling from a distribution close to the one of interest. It
accomplishes this by defining a Markov-Chain whose transition function 
is guaranteed to make it converge asymptotically to the distribution of interest,
given some constraints (ergodicity...) (TODO: explain more)

\subsection{Variational Methods}
\label{subsection:variational}
Variational methods work by turning the problem of integration into one of
optimization. They propose a family of parametric distributions, and then
optimize the parameters so as to minimize the "distance" between the approximate
(normally called "variational") distribution and the distribution of interest.

There are two ways to derive the most commonly used objective function for
this problem.

\subsubsection{Kullback-Leibler Divergence}
\label{subsubsection:kldiv}

The Kullback-Leibler divergence is a measure\footnote{Note that the KL divergence
isn't symmetric and as such I haven't called it a \emph{metric}} of the distance
between two probability distributions $p$, and $q$. It is given by:

\begin{align}
    KL(q||p) = \int q \log\frac{q}{p}
\end{align}

In the setting of inference, $p$ is the posterior $p(z|x)$ and $q$ is a distribution
in some parametric family, with parameters $\phi$, i.e., $q(z; \phi)$. However,
it's clear that we can't compute the Kullback-Leibler directly, because it
requires the knowledge of both the distributions, and finding $p(z|x)$ is precisely
the task at hand. Let us expand the KL divergence expression:

\begin{align}
    KL(q||p) &= \int q(z) (\log q(x) - \log p(z|x)) dz \\
             &= \int q(z) (\log q(z) - (\log p(x, z) - \log p(x))) dz \\
             &= \mathbb{E}_q [\log q(z)] - \mathbb{E}_q [\log p(x, z)] + \mathbb{E}_q [\log p(x)] \\
             &= \mathbb{E}_q [\log q(z)] - \mathbb{E}_q [\log p(x, z)] + \log p(x)
\end{align}

The last term is constant w.r.t $q(z)$. In that sense, for a fixed $p(x)$,
minimizing the KL divergence is equivalent to minimizing
$\mathbb{E}_q [\log q(z)] - \mathbb{E}_q [\log p(x, z)]$, which is equivalent
to maximizing $\mathbb{E}_q [\log p(x, z)] - \mathbb{E}_q [\log q(z)]$. This
quantity is commonly refered to as ELBO - Expectation Lower BOund. It can be
rewritten as:

\begin{align}
    ELBO(q) &= \mathbb{E}_q [\log p(x, z)] - \mathbb{E}_q [\log q(z)] \\
            &= \mathbb{E}_q [\log p(x|z)] + \mathbb{E}_q [\log p(x|z)] - \mathbb{E}_q [\log q(z)]
\end{align}

In this form, each term of the ELBO has an easily interpretable role:
\begin{itemize}
    \item $\mathbb{E}_q [\log p(x|z)]$ tries to maximize the conditional likelihood of $x$. That
        can be seen as assigning high probability mass to values of $z$ that \emph{explain} $x$
        well.
    \item $\mathbb{E}_q [\log p(x|z)]$ is the symmetric of the crossentropy between
        $q(z)$ and $p(x|z)$. Maximizing this quantity is equivalent of maximizing
        that crossentropy. This can be regarded as a regularizer that discourages
        $q(z)$ of being too different from the prior $p(z)$.
    \item $ - \mathbb{E}_q [\log q(z)]$ is the entropy of $q(z)$. Maximizing
        this term incentivizes the probability mass of $q(z)$ to be spread out:
        another form of regularization.
\end{itemize}

\subsubsection{A lower bound on $\log p(x)$}
\label{subsubsection:elbo}

Another way of approaching the intractable posterior is to start by stating
that our inherent goal is to maximize $p(x)$, or equivalently $\log p(x)$. Given
that, consider the following:

\begin{align}
    \log p(x) &= \log \int p(x, z) dz \Leftrightarrow \\
    \Leftrightarrow \log p(x) &= \log \int q(z) \frac{p(x, z)}{q(z)} dz \Leftrightarrow \\
    \Leftrightarrow \log p(x) &= \log \mathbb{E}_q[\frac{p(x, z)}{q(z)}] \Leftrightarrow \\
    \Leftrightarrow \log p(x) &\geq \mathbb{E}_q[\log \frac{p(x, z)}{q(z)}] \Leftrightarrow \\
    \Leftrightarrow \log p(x) &\geq \mathbb{E}_q[\log p(x, z)] - \mathbb{E}_q[q(z)] \Leftrightarrow
\end{align}

We see that the right-hand side of the last inequality is the same quantity
we arrived at in the previous approach, and that it is a lower-bound on the
quantity we want to maximize, and so we want to maximize it. It's worth noting
that when $q(z) = p(z|x)$, the bound is tight.
