% ----------------------------------------------------------------------
% Define document language.
% ----------------------------------------------------------------------

% 'inputenc' package
%
% Accept different input encodings.
% http://www.ctan.org/tex-archive/macros/latex/base/
%
% > allows typing non-english text in LaTeX sources.
%
% ******************************* SELECT *******************************
%\usepackage[latin1]{inputenc} % <<<<< Windows
\usepackage[utf8]{inputenc}   % <<<<< Linux
% ******************************* SELECT *******************************


% 'babel' package
%
% Multilingual support for Plain TeX or LaTeX.
% http://www.ctan.org/tex-archive/macros/latex/required/babel/
%
% > sets the variable names according to the language selected
%
% ******************************* SELECT *******************************
%\usepackage[portuguese]{babel} % <<<<< Portuguese
\usepackage[english]{babel} % <<<<< English
% ******************************* SELECT *******************************


% List of LaTeX variable names: \abstractname, \appendixname, \bibname,
%   \chaptername, \contentsname, \listfigurename, \listtablename, ...)
% http://www.tex.ac.uk/cgi-bin/texfaq2html?label=fixnam
%
\usepackage[acronym]{glossaries}
% Changing the words babel uses (uncomment and redefine as necessary...)
%
\newcommand{\acknowledgments}{@undefined} % new LaTeX variable name
%
% > English

%\addto\captionsenglish{\acronymname}
\addto\captionsenglish{\renewcommand{\acknowledgments}{Acknowledgments}}
%\addto\captionsenglish{\renewcommand{\contentsname}{Contents}}
\addto\captionsenglish{\renewcommand{\listtablename}{List of Tables}}
\addto\captionsenglish{\renewcommand{\listfigurename}{List of Figures}}
%\addto\captionsenglish{\renewcommand{\nomname}{Nomenclature}}
%\addto\captionsenglish{\renewcommand{\glossaryname}{Glossary}}
\addto\captionsenglish{\renewcommand{\acronymname}{List of Acronyms}}
%\addto\captionsenglish{\renewcommand{\bibname}{References}} % Bibliography
%\addto\captionsenglish{\renewcommand{\appendixname}{Appendix}}

% > Portuguese
%
\addto\captionsportuguese{\renewcommand{\acknowledgments}{Agradecimentos}}
%\addto\captionsportuguese{\renewcommand{\contentsname}{Conte\'{u}do}}
%\addto\captionsportuguese{\renewcommand{\listtablename}{Lista de Figuras}}
%\addto\captionsportuguese{\renewcommand{\listfigurename}{Lista de Tabelas}}
%\addto\captionsportuguese{\renewcommand{\nomname}{Lista de S\'{i}mbolos}} % Nomenclatura
%\addto\captionsportuguese{\renewcommand{\glossary}{Gloss\'{a}rio}}
%\addto\captionsportuguese{\renewcommand{\acronymname}{Lista de Abrevia\c{c}\~{o}es}}
%\addto\captionsportuguese{\renewcommand{\bibname}{Refer\^{e}ncias}} % Bibliografia
%\addto\captionsportuguese{\renewcommand{\appendixname}{Anexo}} % Apendice


% ----------------------------------------------------------------------
% Define cover fields in both english and portuguese.
% ----------------------------------------------------------------------
%
\newcommand{\coverThesis}{@undefined} % new LaTeX variable name
\newcommand{\coverSupervisors}{@undefined} % new LaTeX variable name
\newcommand{\coverExaminationCommittee}{@undefined} % new LaTeX variable name
\newcommand{\coverChairperson}{@undefined} % new LaTeX variable name
\newcommand{\coverSupervisor}{@undefined} % new LaTeX variable name
\newcommand{\coverMemberCommittee}{@undefined} % new LaTeX variable name
% > English
\addto\captionsenglish{\renewcommand{\coverThesis}{Thesis to obtain the Master of Science Degree in}}
\addto\captionsenglish{\renewcommand{\coverSupervisors}{Supervisor(s)}}
\addto\captionsenglish{\renewcommand{\coverExaminationCommittee}{Examination Committee}}
\addto\captionsenglish{\renewcommand{\coverChairperson}{Chairperson}}
\addto\captionsenglish{\renewcommand{\coverSupervisor}{Supervisor}}
\addto\captionsenglish{\renewcommand{\coverMemberCommittee}{Member of the Committee}}
% > Portuguese
\addto\captionsportuguese{\renewcommand{\coverThesis}{Disserta\c{c}\~{a}o para obten\c{c}\~{a}o do Grau de Mestre em}}
\addto\captionsportuguese{\renewcommand{\coverSupervisors}{Orientador(es)}}
\addto\captionsportuguese{\renewcommand{\coverExaminationCommittee}{J\'{u}ri}}
\addto\captionsportuguese{\renewcommand{\coverChairperson}{Presidente}}
\addto\captionsportuguese{\renewcommand{\coverSupervisor}{Orientador}}
\addto\captionsportuguese{\renewcommand{\coverMemberCommittee}{Vogal}}


% ----------------------------------------------------------------------
% Define default and cover page fonts.
% ----------------------------------------------------------------------

% Use Arial font as default
%
\renewcommand{\rmdefault}{phv}
\renewcommand{\sfdefault}{phv}

% Define cover page fonts
%
%         encoding     family       series      shape
%  \usefont{T1}     {phv}=helvetica  {b}=bold    {n}=normal
%                   {ptm}=times      {m}=normal  {sl}=slanted
%                                                {it}=italic
% see more examples at
% http://julien.coron.free.fr/languages/latex/fonts/
%
\def\FontLn{% 16 pt normal
  \usefont{T1}{phv}{m}{n}\fontsize{16pt}{16pt}\selectfont}
\def\FontLb{% 16 pt bold
  \usefont{T1}{phv}{b}{n}\fontsize{16pt}{16pt}\selectfont}
\def\FontMn{% 14 pt normal
  \usefont{T1}{phv}{m}{n}\fontsize{14pt}{14pt}\selectfont}
\def\FontMb{% 14 pt bold
  \usefont{T1}{phv}{b}{n}\fontsize{14pt}{14pt}\selectfont}
\def\FontSn{% 12 pt normal
  \usefont{T1}{phv}{m}{n}\fontsize{12pt}{12pt}\selectfont}


% ----------------------------------------------------------------------
% Define page margins and line spacing.
% ----------------------------------------------------------------------

% 'geometry' package
%
% Flexible and complete interface to document dimensions.
% http://www.ctan.org/tex-archive/macros/latex/contrib/geometry/
%
% > set the page margins (2.5cm minimum in every side, as per IST rules)
%
\usepackage{geometry}	
\geometry{verbose,tmargin=2.5cm,bmargin=2.5cm,lmargin=2.5cm,rmargin=2.5cm}

% 'setspace' package
%
% Set space between lines.
% http://www.ctan.org/tex-archive/macros/latex/contrib/setspace/
%
% > allow setting line spacing (line spacing of 1.5, as per IST rules)
%
\usepackage{setspace}
\renewcommand{\baselinestretch}{1.5}


% ----------------------------------------------------------------------
% Include external packages.
% Note that not all of these packages may be available on all system
% installations. If necessary, include the .sty files locally in
% the <jobname>.tex file directory.
% ----------------------------------------------------------------------

% 'graphicx' package
%
% Enhanced support for graphics.
% http://www.ctan.org/tex-archive/macros/latex/required/graphics/
%
% > extends arguments of the \includegraphics command
%
\usepackage{graphicx}


% 'color' package
%
% Colour control for LaTeX documents.
% http://www.ctan.org/tex-archive/macros/latex/required/graphics/
%
% > defines color macros: \color{<color name>}
%
%\usepackage{color}


% 'amsmath' package
%
% Mathematical enhancements for LaTeX.
% http://www.ctan.org/tex-archive/macros/latex/required/amslatex/
%
% > American Mathematical Society plain Tex macros
%
%\usepackage{amsmath}  % AMS mathematical facilities for LaTeX.
\usepackage{mathtools}
\usepackage{amsthm}   % Typesetting theorems (AMS style).
\usepackage{amsfonts} % 


% 'wrapfig' package
%
% Produces figures which text can flow around.
% http://www.ctan.org/tex-archive/macros/latex/contrib/wrapfig/
%
% > wrap figures/tables in text (i.e., Di Vinci style)
%
% \usepackage{wrapfig}


% 'subfigure' package
%
% Deprecated: figures divided into subfigures.
% http://www.ctan.org/tex-archive/obsolete/macros/latex/contrib/subfigure/
%
% > subcaptions for subfigures
%
\usepackage{subfigure}


% 'subfigmat' package
%
% Automates layout when using the subfigure package.
% http://www.ctan.org/tex-archive/macros/latex/contrib/subfigmat/
%
% > matrices of similar subfigures
%
\usepackage{subfigmat}


% 'url' package
%
% Verbatim with URL-sensitive line breaks.
% http://www.ctan.org/tex-archive/macros/latex/contrib/url/
%
% > URLs in BibTex
%
% \usepackage{url}


% 'varioref' package
%
% Intelligent page references.
% http://www.ctan.org/tex-archive/macros/latex/required/tools/
%
% > smart page, figure, table and equation referencing
%
%\usepackage{varioref}


% 'dcolumn' package
%
% Align on the decimal point of numbers in tabular columns.
% http://www.ctan.org/tex-archive/macros/latex/required/tools/
%
% > decimal-aligned tabular math columns
%
\usepackage{dcolumn}
\newcolumntype{d}{D{.}{.}{-1}} % column aligned by the point separator '.'
\newcolumntype{e}{D{E}{E}{-1}} % column aligned by the exponent 'E'


% 'verbatim' package
%
% Reimplementation of and extensions to LaTeX verbatim.
% http://www.ctan.org/tex-archive/macros/latex/required/tools/
%
% > provides the verbatim environment (\begin{verbatim},\end{verbatim})
%   and a comment environment (\begin{comment},  \end{comment})
%
% \usepackage{verbatim}


% 'moreverb' package
%
% Extended verbatim.
% http://www.ctan.org/tex-archive/macros/latex/contrib/moreverb/
%
% > supports tab expansion and line numbering
%
% \usepackage{moreverb}



% 'nomencl' package
%
% Produce lists of symbols as in nomenclature.
% http://www.ctan.org/tex-archive/macros/latex/contrib/nomencl/
%
% The nomencl package makes use of the MakeIndex program
% in order to produce the nomenclature list.
%
% Nomenclature
% 1) On running the file through LATEX, the command \makenomenclature
%    in the preamble instructs it to create/open the nomenclature file
%    <jobname>.nlo corresponding to the LATEX file <jobname>.tex and
%    writes the information from the \nomenclature commands to this file.
% 2) The next step is to invoke MakeIndex in order to produce the
%    <jobname>.nls file. This can be achieved by making use of the
%    command: makeindex <jobname>.nlo -s nomencl.ist -o <jobname>.nls
% 3) The last step is to invoke LATEX on the <jobname>.tex file once
%    more. There, the \printnomenclature in the document will input the
%    <jobname>.nls file and process it according to the given options.
%
% http://www-h.eng.cam.ac.uk/help/tpl/textprocessing/nomencl.pdf
%
% Nomenclature (produces *.nlo *.nls files)
\usepackage{nomencl}
\makenomenclature
%
% Group variables according to their symbol type
%
\RequirePackage{ifthen} 
\ifthenelse{\equal{\languagename}{english}}%
    { % English
    \renewcommand{\nomgroup}[1]{%
      \ifthenelse{\equal{#1}{R}}{%
        \item[\textbf{Roman symbols}]}{%
        \ifthenelse{\equal{#1}{G}}{%
          \item[\textbf{Greek symbols}]}{%
          \ifthenelse{\equal{#1}{S}}{%
            \item[\textbf{Subscripts}]}{%
            \ifthenelse{\equal{#1}{T}}{%
              \item[\textbf{Superscripts}]}{}}}}}%
    }{% Portuguese
    \renewcommand{\nomgroup}[1]{%
      \ifthenelse{\equal{#1}{R}}{%
        \item[\textbf{Simbolos romanos}]}{%
        \ifthenelse{\equal{#1}{G}}{%
          \item[\textbf{Simbolos gregos}]}{%
          \ifthenelse{\equal{#1}{S}}{%
            \item[\textbf{Subscritos}]}{%
            \ifthenelse{\equal{#1}{T}}{%
              \item[\textbf{Sobrescritos}]}{}}}}}%
    }%


% 'glossary' package
%
% Create a glossary.
% http://www.ctan.org/tex-archive/macros/latex/contrib/glossary/
%
%% Glossary (produces *.glo *.ist files)
%\usepackage[number=none]{glossary}
%% (remove blank line between groups)
%\setglossary{gloskip={}}
%% (redefine glossary style file)
%%\renewcommand{\istfilename}{myGlossaryStyle.ist}
%\makeglossary


% 'rotating' package
%
% Rotation tools, including rotated full-page floats.
% http://www.ctan.org/tex-archive/macros/latex/contrib/rotating/
%
% > show wide figures and tables in landscape format:
%   use \begin{sidewaystable} and \begin{sidewaysfigure}
%   instead of 'table' and 'figure', respectively.
%
\usepackage{rotating}


% 'hyperref' package
%
% Extensive support for hypertext in LaTeX.
% http://www.ctan.org/tex-archive/macros/latex/contrib/hyperref/
%
% > Extends the functionality of all the LATEX cross-referencing
%   commands (including the table of contents, bibliographies etc) to
%   produce \special commands which a driver can turn into hypertext
%   links; Also provides new commands to allow the user to write adhoc
%   hypertext links, including those to external documents and URLs.
%
\usepackage[pdftex]{hyperref} % enhance documents that are to be
                              % output as HTML and PDF
\hypersetup{colorlinks,       % color text of links and anchors,
                              % eliminates borders around links
%            linkcolor=red,    % color for normal internal links
            linkcolor=black,  % color for normal internal links
            anchorcolor=black,% color for anchor text
%            citecolor=green,  % color for bibliographical citations
            citecolor=black,  % color for bibliographical citations
%            filecolor=magenta,% color for URLs which open local files
            filecolor=black,  % color for URLs which open local files
%            menucolor=red,    % color for Acrobat menu items
            menucolor=black,  % color for Acrobat menu items
%            pagecolor=red,    % color for links to other pages
            pagecolor=black,  % color for links to other pages
%            urlcolor=cyan,    % color for linked URLs
            urlcolor=black,   % color for linked URLs
	          bookmarks=true,         % create PDF bookmarks
	          bookmarksopen=false,    % don't expand bookmarks
	          bookmarksnumbered=true, % number bookmarks
	          pdftitle={Thesis},
            pdfauthor={Guilherme Pires},
            pdfsubject={VMoNF},
            pdfstartview=FitV,
            pdfdisplaydoctitle=true}


% 'hypcap' package
%
% Adjusting the anchors of captions.
% http://www.ctan.org/tex-archive/macros/latex/contrib/oberdiek/
%
% > fixes the problem with hyperref, that links to floats points
%   below the caption and not at the beginning of the float.
%
\usepackage[figure,table]{hypcap}


\usepackage[
    backend=biber,
    style=authoryear,
    sortlocale=en_EN,
    natbib=true,
    url=false,
    doi=true,
    eprint=false
]{biblatex}
\newrobustcmd*{\parentexttrack}[1]{%
  \begingroup
  \blx@blxinit
  \blx@setsfcodes
  \blx@bibopenparen#1\blx@bibcloseparen
  \endgroup}

\AtEveryCite{%
  \let\parentext=\parentexttrack%
  \let\bibopenparen=\bibopenbracket%
  \let\bibcloseparen=\bibclosebracket}

\makeatother
\addbibresource{thesis.bib}
% ******************************* SELECT *******************************


% 'notoccite' package
%
% Prevent trouble from citations in table of contents, etc.
% http://ctan.org/pkg/notoccite
%
% > If you have \cite com­mands in \sec­tion-like com­mands, or in \cap­tion,
%   the ci­ta­tion will also ap­pear in the ta­ble of con­tents, or list of what­ever.
%   If you are also us­ing an un­srt-like bib­li­og­ra­phy style, these ci­ta­tions will
%   come at the very start of the bib­li­og­ra­phy, which is con­fus­ing. This pack­age
%   sup­presses the ef­fect.
%
\usepackage{notoccite}


% 'multirow' package
%
% Create tabular cells spanning multiple rows
% http://www.ctan.org/pkg/multirow
%
\usepackage{multirow}


% 'booktabs' package
%
% Publication quality tables in LaTeX
% http://www.ctan.org/pkg/booktabs
%
% > en­hance the qual­ity of ta­bles in LaTeX, pro­vid­ing ex­tra com­mands.
%
% \renewcommand{\arraystretch}{<ratio>} % space between rows
%
\usepackage{booktabs}
%\newcommand{\ra}[1]{\renewcommand{\arraystretch}{#1}}


% 'pdfpages' package
%
% Include PDF documents in LaTeX
% http://www.ctan.org/pkg/pdfpages
%
% > in­clu­sion of ex­ter­nal multi-page PDF doc­u­ments in LaTeX doc­u­ments.
%   Pages may be freely se­lected and sim­i­lar to psnup it is pos­si­ble to put
%   sev­eral log­i­cal pages onto each sheet of pa­per.
%
% \includepdf{filename.pdf}
% \includepdf[pages={4-9},nup=2x3,landscape=true]{filename.pdf}
%
\usepackage{pdfpages}

\usepackage{enumitem}
\usepackage{arydshln}
\usepackage{tikz}
\setlist{nosep}

% ----------------------------------------------------------------------
% Define new commands to assure consistent treatment throughout document
% ----------------------------------------------------------------------

\newcommand{\ud}{\mathrm{d}}                % total derivative
\newcommand{\degree}{\ensuremath{^\circ\,}} % degrees

% Abbreviations

\newcommand{\mcol}{\multicolumn}            % table format

\newcommand{\eqnref}[1]{(\ref{#1})}
\newcommand{\class}[1]{\texttt{#1}}
\newcommand{\package}[1]{\texttt{#1}}
\newcommand{\file}[1]{\texttt{#1}}
\newcommand{\BibTeX}{\textsc{Bib}\TeX}

% Typefaces ( example: {\bf Bold text here} )
%
% > pre-defined
%   \bf % bold face
%   \it % italic
%   \tt % typewriter
%
% > newly defined
\newcommand{\tr}[1]{{\ensuremath{\textrm{#1}}}}   % text roman
\newcommand{\tb}[1]{{\ensuremath{\textbf{#1}}}}   % text bold face
\newcommand{\ti}[1]{{\ensuremath{\textit{#1}}}}   % text italic
\newcommand{\mc}[1]{{\ensuremath{\mathcal{#1}}}}  % math calygraphy
\newcommand{\mco}[1]{{\ensuremath{\mathcalold{#1}}}}% math old calygraphy
\newcommand{\mr}[1]{{\ensuremath{\mathrm{#1}}}}   % math roman
\newcommand{\mb}[1]{{\ensuremath{\mathbf{#1}}}}   % math bold face
\newcommand{\bs}[1]{\ensuremath{\boldsymbol{#1}}} % math symbol
\def\bm#1{\mathchoice                             % math bold
  {\mbox{\boldmath$\displaystyle#1$}}%
  {\mbox{\boldmath$#1$}}%
  {\mbox{\boldmath$\scriptstyle#1$}}%
  {\mbox{\boldmath$\scriptscriptstyle#1$}}}
\newcommand{\boldcal}[1]{{\ensuremath{\boldsymbol{\mathcal{#1}}}}}% math bold calygraphy
\DeclareMathOperator*{\argmin}{\arg\!\min}
\DeclareMathOperator*{\argmax}{\arg\!\max}

\usepackage{tikz}
\usetikzlibrary{matrix,decorations.pathreplacing,calc}
\begin{acronym}
\acro{acro}{Dummy Acronym}
\end{acronym}


\usepackage{bm}
%\usepackage[]{algorithm2e}

\usepackage{color}
\usepackage{listings}
\usepackage{caption}

\newcounter{nalg}[chapter] % defines algorithm counter for chapter-level
\renewcommand{\thenalg}{\thechapter .\arabic{nalg}} %defines appearance of the algorithm counter
\DeclareCaptionLabelFormat{algocaption}{Algorithm \thenalg} % defines a new caption label as Algorithm x.y

\lstnewenvironment{algorithm}[1][] %defines the algorithm listing environment
{
    \refstepcounter{nalg} %increments algorithm number
    \captionsetup{labelformat=algocaption,labelsep=colon} %defines the caption setup for: it ises label format as the declared caption label above and makes label and caption text to be separated by a ':'
    \lstset{ %this is the stype
        mathescape=true,
        frame=single,
        captionpos=b,
        aboveskip=25pt,
        belowskip=25pt,
        numbers=none,
        keywordstyle=\color{black}\bfseries,
        keywords={for, input, output, return, datatype, function, in, if, else, foreach, while, begin, end, } %add the keywords you want, or load a language as Rubens explains in his comment above.
        xleftmargin=.04\textwidth,
        #1 % this is to add specific settings to an usage of this environment (for instnce, the caption and referable label)
    }
}
{}

\newcommand{\q}[1]{``#1''} 
\usepackage{diagbox}
