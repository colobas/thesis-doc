\usepackage[utf8]{inputenc}   % <<<<< Linux
\usepackage[english]{babel} % <<<<< English
\newcommand{\acknowledgments}{@undefined} % new LaTeX variable name
\addto\captionsenglish{\renewcommand{\acknowledgments}{Acknowledgments}}
%\addto\captionsenglish{\renewcommand{\contentsname}{Contents}}
\addto\captionsenglish{\renewcommand{\listtablename}{List of Tables}}
\addto\captionsenglish{\renewcommand{\listfigurename}{List of Figures}}
%\addto\captionsenglish{\renewcommand{\nomname}{Nomenclature}}
%\addto\captionsenglish{\renewcommand{\bibname}{References}} % Bibliography
%\addto\captionsenglish{\renewcommand{\appendixname}{Appendix}}

% > Portuguese
%
\addto\captionsportuguese{\renewcommand{\acknowledgments}{Agradecimentos}}
%\addto\captionsportuguese{\renewcommand{\contentsname}{Conte\'{u}do}}
%\addto\captionsportuguese{\renewcommand{\listtablename}{Lista de Figuras}}
%\addto\captionsportuguese{\renewcommand{\listfigurename}{Lista de Tabelas}}
%\addto\captionsportuguese{\renewcommand{\nomname}{Lista de S\'{i}mbolos}} % Nomenclatura
%\addto\captionsportuguese{\renewcommand{\bibname}{Refer\^{e}ncias}} % Bibliografia
%\addto\captionsportuguese{\renewcommand{\appendixname}{Anexo}} % Apendice


\usepackage{graphicx}


% 'color' package
%
% Colour control for LaTeX documents.
% http://www.ctan.org/tex-archive/macros/latex/required/graphics/
%
% > defines color macros: \color{<color name>}
%
%\usepackage{color}


% 'amsmath' package
%
% Mathematical enhancements for LaTeX.
% http://www.ctan.org/tex-archive/macros/latex/required/amslatex/
%
% > American Mathematical Society plain Tex macros
%
%\usepackage{amsmath}  % AMS mathematical facilities for LaTeX.
\usepackage{mathtools}
\usepackage{amsthm}   % Typesetting theorems (AMS style).
\usepackage{amsfonts} % 


% 'wrapfig' package
%
% Produces figures which text can flow around.
% http://www.ctan.org/tex-archive/macros/latex/contrib/wrapfig/
%
% > wrap figures/tables in text (i.e., Di Vinci style)
%
% \usepackage{wrapfig}


% 'subfigure' package
%
% Deprecated: figures divided into subfigures.
% http://www.ctan.org/tex-archive/obsolete/macros/latex/contrib/subfigure/
%
% > subcaptions for subfigures
%
\usepackage{subfigure}


% 'subfigmat' package
%
% Automates layout when using the subfigure package.
% http://www.ctan.org/tex-archive/macros/latex/contrib/subfigmat/
%
% > matrices of similar subfigures
%
\usepackage{subfigmat}


% 'url' package
%
% Verbatim with URL-sensitive line breaks.
% http://www.ctan.org/tex-archive/macros/latex/contrib/url/
%
% > URLs in BibTex
%
% \usepackage{url}


% 'varioref' package
%
% Intelligent page references.
% http://www.ctan.org/tex-archive/macros/latex/required/tools/
%
% > smart page, figure, table and equation referencing
%
%\usepackage{varioref}


% 'dcolumn' package
%
% Align on the decimal point of numbers in tabular columns.
% http://www.ctan.org/tex-archive/macros/latex/required/tools/
%
% > decimal-aligned tabular math columns
%
\usepackage{dcolumn}
\newcolumntype{d}{D{.}{.}{-1}} % column aligned by the point separator '.'
\newcolumntype{e}{D{E}{E}{-1}} % column aligned by the exponent 'E'


% 'verbatim' package
%
% Reimplementation of and extensions to LaTeX verbatim.
% http://www.ctan.org/tex-archive/macros/latex/required/tools/
%
% > provides the verbatim environment (\begin{verbatim},\end{verbatim})
%   and a comment environment (\begin{comment},  \end{comment})
%
% \usepackage{verbatim}


% 'moreverb' package
%
% Extended verbatim.
% http://www.ctan.org/tex-archive/macros/latex/contrib/moreverb/
%
% > supports tab expansion and line numbering
%
% \usepackage{moreverb}



% 'nomencl' package
%
% Produce lists of symbols as in nomenclature.
% http://www.ctan.org/tex-archive/macros/latex/contrib/nomencl/
%
% The nomencl package makes use of the MakeIndex program
% in order to produce the nomenclature list.
%
% Nomenclature
% 1) On running the file through LATEX, the command \makenomenclature
%    in the preamble instructs it to create/open the nomenclature file
%    <jobname>.nlo corresponding to the LATEX file <jobname>.tex and
%    writes the information from the \nomenclature commands to this file.
% 2) The next step is to invoke MakeIndex in order to produce the
%    <jobname>.nls file. This can be achieved by making use of the
%    command: makeindex <jobname>.nlo -s nomencl.ist -o <jobname>.nls
% 3) The last step is to invoke LATEX on the <jobname>.tex file once
%    more. There, the \printnomenclature in the document will input the
%    <jobname>.nls file and process it according to the given options.
%
% http://www-h.eng.cam.ac.uk/help/tpl/textprocessing/nomencl.pdf
%
% Nomenclature (produces *.nlo *.nls files)
\usepackage{nomencl}
\makenomenclature
%
% Group variables according to their symbol type
%
\RequirePackage{ifthen} 
\ifthenelse{\equal{\languagename}{english}}%
    { % English
    \renewcommand{\nomgroup}[1]{%
      \ifthenelse{\equal{#1}{R}}{%
        \item[\textbf{Roman symbols}]}{%
        \ifthenelse{\equal{#1}{G}}{%
          \item[\textbf{Greek symbols}]}{%
          \ifthenelse{\equal{#1}{S}}{%
            \item[\textbf{Subscripts}]}{%
            \ifthenelse{\equal{#1}{T}}{%
              \item[\textbf{Superscripts}]}{}}}}}%
    }{% Portuguese
    \renewcommand{\nomgroup}[1]{%
      \ifthenelse{\equal{#1}{R}}{%
        \item[\textbf{Simbolos romanos}]}{%
        \ifthenelse{\equal{#1}{G}}{%
          \item[\textbf{Simbolos gregos}]}{%
          \ifthenelse{\equal{#1}{S}}{%
            \item[\textbf{Subscritos}]}{%
            \ifthenelse{\equal{#1}{T}}{%
              \item[\textbf{Sobrescritos}]}{}}}}}%
    }%


\usepackage{rotating}
\usepackage[pdftex]{hyperref} % enhance documents that are to be
                              % output as HTML and PDF
\hypersetup{bookmarks=true,         % create PDF bookmarks
            bookmarksopen=false,    % don't expand bookmarks
            bookmarksnumbered=true, % number bookmarks
            pdftitle={Thesis},
            pdfauthor={Guilherme Pires},
            pdfsubject={VMoNF},
            pdfstartview=FitV,
            pdfdisplaydoctitle=true}

\usepackage[figure,table]{hypcap}
\usepackage[
    backend=biber,
    style=ieee,
    sortlocale=en_EN,
    natbib=true,
    url=false,
    doi=true,
    eprint=false
]{biblatex}
\newrobustcmd*{\parentexttrack}[1]{%
  \begingroup
  \blx@blxinit
  \blx@setsfcodes
  \blx@bibopenparen#1\blx@bibcloseparen
  \endgroup}

\AtEveryCite{%
  \let\parentext=\parentexttrack%
  \let\bibopenparen=\bibopenbracket%
  \let\bibcloseparen=\bibclosebracket}

\makeatother
\addbibresource{arxiv.bib}

\usepackage{notoccite}

\usepackage{multirow}

\usepackage{booktabs}

\usepackage{pdfpages}

\usepackage{enumitem}
\usepackage{arydshln}
\usepackage{tikz}
\setlist{nosep}

\newcommand{\ud}{\mathrm{d}}                % total derivative
\newcommand{\degree}{\ensuremath{^\circ\,}} % degrees

\newcommand{\mcol}{\multicolumn}            % table format

\newcommand{\eqnref}[1]{(\ref{#1})}
\newcommand{\class}[1]{\texttt{#1}}
\newcommand{\package}[1]{\texttt{#1}}
\newcommand{\file}[1]{\texttt{#1}}
\newcommand{\BibTeX}{\textsc{Bib}\TeX}
\newcommand{\tr}[1]{{\ensuremath{\textrm{#1}}}}   % text roman
\newcommand{\tb}[1]{{\ensuremath{\textbf{#1}}}}   % text bold face
\newcommand{\ti}[1]{{\ensuremath{\textit{#1}}}}   % text italic
\newcommand{\mc}[1]{{\ensuremath{\mathcal{#1}}}}  % math calygraphy
\newcommand{\mco}[1]{{\ensuremath{\mathcalold{#1}}}}% math old calygraphy
\newcommand{\mr}[1]{{\ensuremath{\mathrm{#1}}}}   % math roman
\newcommand{\mb}[1]{{\ensuremath{\mathbf{#1}}}}   % math bold face
\newcommand{\bs}[1]{\ensuremath{\boldsymbol{#1}}} % math symbol
\def\bm#1{\mathchoice                             % math bold
  {\mbox{\boldmath$\displaystyle#1$}}%
  {\mbox{\boldmath$#1$}}%
  {\mbox{\boldmath$\scriptstyle#1$}}%
  {\mbox{\boldmath$\scriptscriptstyle#1$}}}
\newcommand{\boldcal}[1]{{\ensuremath{\boldsymbol{\mathcal{#1}}}}}% math bold calygraphy
\DeclareMathOperator*{\argmin}{\arg\!\min}
\DeclareMathOperator*{\argmax}{\arg\!\max}

\usepackage{tikz}
\usetikzlibrary{matrix,decorations.pathreplacing,calc}

\usepackage{bm}
%\usepackage[]{algorithm2e}

\usepackage{color}
\usepackage{listings}
\usepackage{caption}

\lstnewenvironment{algorithm}[1][] %defines the algorithm listing environment
{
    \refstepcounter{nalg} %increments algorithm number
    \captionsetup{labelformat=algocaption,labelsep=colon} %defines the caption setup for: it ises label format as the declared caption label above and makes label and caption text to be separated by a ':'
    \lstset{ %this is the stype
        mathescape=true,
        frame=single,
        captionpos=b,
        aboveskip=25pt,
        belowskip=25pt,
        numbers=none,
        keywordstyle=\color{black}\bfseries,
        keywords={for, input, output, return, datatype, function, in, if, else, foreach, while, begin, end, } %add the keywords you want, or load a language as Rubens explains in his comment above.
        xleftmargin=.04\textwidth,
        #1 % this is to add specific settings to an usage of this environment (for instnce, the caption and referable label)
    }
}
{}

\newcommand{\q}[1]{``#1''} 

% correct bad hyphenation here
\hyphenation{op-tical net-works semi-conduc-tor}
\usepackage{diagbox}
