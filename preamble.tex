\usepackage[utf8]{inputenc}   % <<<<< Linux
\usepackage[english]{babel} % <<<<< English
\usepackage[dvips]{graphicx}
\usepackage{mathtools}
\mathtoolsset{showonlyrefs}
\usepackage{amsthm}   % Typesetting theorems (AMS style).
\usepackage{amsfonts} % 
\usepackage{dcolumn}
\newcolumntype{d}{D{.}{.}{-1}} % column aligned by the point separator '.'
\newcolumntype{e}{D{E}{E}{-1}} % column aligned by the exponent 'E'
\usepackage{rotating}
\usepackage[pdftex]{hyperref} % enhance documents that are to be
                              % output as HTML and PDF
\hypersetup{colorlinks,       % color text of links and anchors,
                              % eliminates borders around links
%            linkcolor=red,    % color for normal internal links
            linkcolor=black,  % color for normal internal links
            anchorcolor=black,% color for anchor text
%            citecolor=green,  % color for bibliographical citations
            citecolor=black,  % color for bibliographical citations
%            filecolor=magenta,% color for URLs which open local files
            filecolor=black,  % color for URLs which open local files
%            menucolor=red,    % color for Acrobat menu items
            menucolor=black,  % color for Acrobat menu items
%            pagecolor=red,    % color for links to other pages
            pagecolor=black,  % color for links to other pages
%            urlcolor=cyan,    % color for linked URLs
            urlcolor=black,   % color for linked URLs
            bookmarks=true,         % create PDF bookmarks
            bookmarksopen=false,    % don't expand bookmarks
            bookmarksnumbered=true, % number bookmarks
            pdftitle={VMoNF},
            pdfauthor={Guilherme Pires},
            pdfsubject={Variational Mixture of Normalizing Flows},
            pdfkeywords={},
            pdfstartview=FitV,
            pdfdisplaydoctitle=true}
\usepackage[figure,table]{hypcap}
\usepackage[backend=biber,sorting=none,maxnames=1,firstinits=true,doi=false,url=false,natbib=true]{biblatex}

\DeclareFieldFormat
  [article,inbook,incollection,inproceedings,patent,thesis,unpublished]
  {title}{#1}

\renewbibmacro{in:}{%
  \ifentrytype{article}{%
  }{%
    \printtext{\bibstring{in}\intitlepunct}%
  }%
}

\DeclareFieldFormat{pages}{#1}
\DeclareFieldFormat[article]{number}{\mkbibparens{#1}}

\renewbibmacro*{volume+number+eid}{%
  \setunit{\addcomma\space}%  
  \printfield{volume}%
  \printfield{number}%
  \setunit{\addcomma\space}%
  \printfield{eid}%
  \setunit{\addcolon}%
  \printfield{pages}}

\renewbibmacro*{issue+date}{%
  \setunit{\addcomma\space}%
  \iffieldundef{issue}
    {\usebibmacro{date}}
    {\printfield{issue}%
     \setunit*{\addspace}%
     \usebibmacro{date}}%
\newunit}

\renewbibmacro*{note+pages}{%
  \printfield{note}%
  \newunit}

\newrobustcmd*{\parentexttrack}[1]{%
  \begingroup
  \blx@blxinit
  \blx@setsfcodes
  \blx@bibopenparen#1\blx@bibcloseparen
  \endgroup}

\AtEveryCite{%
  \let\parentext=\parentexttrack%
  \let\bibopenparen=\bibopenbracket%
  \let\bibcloseparen=\bibclosebracket}

\makeatother
\addbibresource{esannV2.bib}

\usepackage{notoccite}

\usepackage{multirow}

\usepackage{booktabs}

\usepackage{pdfpages}

\usepackage{enumitem}
\usepackage{arydshln}
\usepackage{tikz}
\setlist{nosep}

\newcommand{\ud}{\mathrm{d}}                % total derivative
\newcommand{\degree}{\ensuremath{^\circ\,}} % degrees

\newcommand{\mcol}{\multicolumn}            % table format

\newcommand{\eqnref}[1]{(\ref{#1})}
\newcommand{\class}[1]{\texttt{#1}}
\newcommand{\package}[1]{\texttt{#1}}
\newcommand{\file}[1]{\texttt{#1}}
\newcommand{\BibTeX}{\textsc{Bib}\TeX}
\newcommand{\tr}[1]{{\ensuremath{\textrm{#1}}}}   % text roman
\newcommand{\tb}[1]{{\ensuremath{\textbf{#1}}}}   % text bold face
\newcommand{\ti}[1]{{\ensuremath{\textit{#1}}}}   % text italic
\newcommand{\mc}[1]{{\ensuremath{\mathcal{#1}}}}  % math calygraphy
\newcommand{\mco}[1]{{\ensuremath{\mathcalold{#1}}}}% math old calygraphy
\newcommand{\mr}[1]{{\ensuremath{\mathrm{#1}}}}   % math roman
\newcommand{\mb}[1]{{\ensuremath{\mathbf{#1}}}}   % math bold face
\newcommand{\bs}[1]{\ensuremath{\boldsymbol{#1}}} % math symbol
\def\bm#1{\mathchoice                             % math bold
  {\mbox{\boldmath$\displaystyle#1$}}%
  {\mbox{\boldmath$#1$}}%
  {\mbox{\boldmath$\scriptstyle#1$}}%
  {\mbox{\boldmath$\scriptscriptstyle#1$}}}
\newcommand{\boldcal}[1]{{\ensuremath{\boldsymbol{\mathcal{#1}}}}}% math bold calygraphy
\DeclareMathOperator*{\argmin}{\arg\!\min}
\DeclareMathOperator*{\argmax}{\arg\!\max}

\usepackage{tikz}
\usetikzlibrary{matrix,decorations.pathreplacing,calc}

\usepackage{bm}
%\usepackage[]{algorithm2e}

\usepackage{color}
\usepackage{listings}
\usepackage{caption}
\usepackage{subcaption}
\newcommand{\q}[1]{``#1''} 
\hyphenation{op-tical net-works semi-conduc-tor}
\usepackage{diagbox}
\usepackage{floatrow}
\newfloatcommand{capbtabbox}{table}[t][\FBwidth]
\usepackage{setspace}
\usepackage{siunitx}

\floatsetup{heightadjust=object}
