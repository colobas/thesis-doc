\section{Normalizing Flows}
\label{cov}
Given a random variable $\bm{z} \in \mathbb{R}^D$, with probability density function $f_Z$,
and a bijective and continuous function $g\: : \: \mathbb{R}^D\rightarrow \mathbb{R}^D$,
the probability density function $f_X$ of the random variable $\bm{x} = g(\bm{z})$ is given by
\begin{align}
    f_X(\bm{x}) &= f_Z(g^{-1}(\bm{x}))\Big|\det\Big(\frac{d}{d\bm{x}}g^{-1}(\bm{x})\Big)\Big|,
\end{align} where $\det\Big(\frac{d}{d\bm{x}}g^{-1}(\bm{x})\Big)$
is the determinant of the Jacobian matrix of $g^{-1}$, computed at $\bm{x}$.
If $g$ is parameterized by some $\bm\theta$,
this expression can be optimized, so that it approximates some arbitrary distribution.
For that to be feasible, the \emph{base density}, $f_Z$, has to be computationally \q{cheap} to evaluate,
as well as $\det\Big(\frac{d}{d\bm{x}}g^{-1}(\bm{x};\bm\theta)\Big)$, and its gradient w.r.t
$\bm\theta$.

A \emph{normalizing flow} - a model proposed in \autocite{shakir_nf}, and which
has since developed into the basis of multiple state-of-the-art techniques for
density estimation \autocites{Glow}{real-nvp}{bnaf19}{maf} - is obtained by
applying the change of variables formula to a function $g$ which is the composition
of $L$ (parametric) transformations $h_\ell$, for ${\ell = 0, 1, ..., L-1}$, which
fulfill the mentioned computational requirements.
Let $\bm{z_0}$ be sampled from $f_Z$. Applying the change of variables formula
to $\bm{x} = g(\bm{z_0})$, and taking the logarithm, yields:
\begin{align}
    \log f_X(\bm{x}) = \log f_Z(g^{-1}(\bm{x})) - \sum_{\ell=0}^{L-1} \log \Big|\det\Big(\frac{d}{d\bm{x_{\ell}}}h_{\ell}(\bm{x_\ell})\Big) \Big|. \label{eq:nflowsfinal}
\end{align}

The design of transformations that are sufficiently expressive, and whose Jacobians
are not computationally heavy, is the main challenge of the framework of normalizing
flows. \textcites{real-nvp}{maf} present examples of such transformations, which
are used in the experiments in this work.

